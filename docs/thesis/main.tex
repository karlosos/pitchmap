%%%%%%%%%%%%%%%%%%%%%%%%%%%%%%%%%%%%%%%%%
% The Legrand Orange Book
% LaTeX Template
% Version 2.4 (26/09/2018)
%
% This template was downloaded from:
% http://www.LaTeXTemplates.com
%
% Original author:
% Mathias Legrand (legrand.mathias@gmail.com) with modifications by:
% Vel (vel@latextemplates.com)
%
% License:
% CC BY-NC-SA 3.0 (http://creativecommons.org/licenses/by-nc-sa/3.0/)
%
% Compiling this template:
% This template uses biber for its bibliography and makeindex for its index.
% When you first open the template, compile it from the command line with the 
% commands below to make sure your LaTeX distribution is configured correctly:
%
% 1) pdflatex main
% 2) makeindex main.idx -s StyleInd.ist
% 3) biber main
% 4) pdflatex main x 2
%
% After this, when you wish to update the bibliography/index use the appropriate
% command above and make sure to compile with pdflatex several times 
% afterwards to propagate your changes to the document.
%
% This template also uses a number of packages which may need to be
% updated to the newest versions for the template to compile. It is strongly
% recommended you update your LaTeX distribution if you have any
% compilation errors.
%
% Important note:
% Chapter heading images should have a 2:1 width:height ratio,
% e.g. 920px width and 460px height.
%
%%%%%%%%%%%%%%%%%%%%%%%%%%%%%%%%%%%%%%%%%
% Template was modified and adopted as a new standard
% for the Department of Computer Science
% Wespomeranian University of Technology 
% by
% Joanna Kolodziejczyk (kolodziejczykj@gmail.com)
% Version 1.0 (10/01/2019)
% Modification assigned by %JK
%%%%%%%%%%%%%%%%%%%%%%%%%%%%%%%%%%%%%%%%%


%----------------------------------------------------------------------------------------
%	PACKAGES AND OTHER DOCUMENT CONFIGURATIONS
%----------------------------------------------------------------------------------------

\documentclass[12pt,fleqn,twoside]{book} % Default font size and left-justified equations 
% JK - font


%%%%%%%%%%%%%%%%%%%%%%%%%%%%%%%%%%%%%%%%%
% The Legrand Orange Book
% Structural Definitions File
% Version 2.1 (26/09/2018)
%
% Original author:
% Mathias Legrand (legrand.mathias@gmail.com) with modifications by:
% Vel (vel@latextemplates.com)
% 
% This file was downloaded from:
% http://www.LaTeXTemplates.com
%
% License:
% CC BY-NC-SA 3.0 (http://creativecommons.org/licenses/by-nc-sa/3.0/)
%
%%%%%%%%%%%%%%%%%%%%%%%%%%%%%%%%%%%%%%%%%
% Template was modified and adopted as a new standard
% for the Department of Computer Science
% Wespomeranian University of Technology 
% by
% Joanna Kolodziejczyk (kolodziejczykj@gmail.com)
% Modification assigned by %JK
%%%%%%%%%%%%%%%%%%%%%%%%%%%%%%%%%%%%%%%%%

%----------------------------------------------------------------------------------------
%	VARIOUS REQUIRED PACKAGES AND CONFIGURATIONS
%----------------------------------------------------------------------------------------


\usepackage{graphicx} % Required for including pictures
\graphicspath{{Pictures/}} % Specifies the directory where pictures are stored

\usepackage{lipsum} % Inserts dummy text

\usepackage{tikz} % Required for drawing custom shapes

\usepackage{amsmath,amsfonts,amssymb,amsthm} % For math equations, theorems, symbols, etc
\usepackage[polish]{babel} % JK- Polish language/hyphenation
% moved after to avoid conflict between polish babel and amsmath

\usepackage{enumitem} % Customize lists
\setlist{nolistsep} % Reduce spacing between bullet points and numbered lists

\usepackage{booktabs} % Required for nicer horizontal rules in tables

\usepackage{xcolor} % Required for specifying colors by name
\definecolor{blueWI}{cmyk}{.6,.2,0,.0} % JK - Define the blue colour used for highlighting throughout the book
\definecolor{blueZUT}{cmyk}{1,.75,0,.2} % JK - Define the blue colour used for highlighting throughout the book
\definecolor{grayZUT}{cmyk}{0,0,0,0.4} % JK - Define the blue colour used for highlighting throughout the book

%other definision
\newcommand{\rulecolor}[1]{\color{#1}\rule}


%----------------------------------------------------------------------------------------
%	MARGINS
%----------------------------------------------------------------------------------------

\usepackage{geometry} % Required for adjusting page dimensions and margins

\geometry{
	paper=a4paper, % Paper size, change to letterpaper for US letter size
	bindingoffset=1cm, %cm for binding
	top=2.5cm, % Top margin
	bottom=2.5cm, % Bottom margin
	left=2.5cm, % Left margin
	right=2.5cm, % Right margin
	headheight=17pt, % Header height
	footskip=1.4cm, % Space from the bottom margin to the baseline of the footer
	headsep=15pt, % Space from the top margin to the baseline of the header
	%showframe, % Uncomment to show how the type block is set on the page
}

%----------------------------------------------------------------------------------------
%	FONTS
%----------------------------------------------------------------------------------------

\usepackage{avant} % Use the Avantgarde font for headings
%\usepackage{times} % Use the Times font for headings
%\usepackage{mathptmx} % Use the Adobe Times Roman as the default text font together with math symbols from the Sym­bol, Chancery and Com­puter Modern fonts

\usepackage{microtype} % Slightly tweak font spacing for aesthetics
\usepackage[utf8]{inputenc} % Required for including letters with accents
\usepackage[T1]{fontenc} % Use 8-bit encoding that has 256 glyphs




%----------------------------------------------------------------------------------------
%	BIBLIOGRAPHY
%----------------------------------------------------------------------------------------
% JK - configuring and styling
%----------------------------------------------------------------------------------------
%=numeric,citestyle
\usepackage[
style=numeric,% style alphabetic or numeric
citestyle=numeric,%similar to style
sorting=nyt,%name -year -title
sortcites=true,
autopunct=true,
autolang=hyphen,
hyperref=true, % if the citation is the link to bibliography
abbreviate=false, %back ref in short or long version
backref=true, % gives the info where the bib was cited (pages)
backend=biber,
defernumbers=true]{biblatex}
\addbibresource{bibliography.bib} % BibTeX bibliography file
\defbibheading{bibempty}{}

\usepackage{calc} % For simpler calculation - used for spacing the index letter headings correctly


%----------------------------------------------------------------------------------------
%	CHAPTER & SECTION HEADINGS
%----------------------------------------------------------------------------------------

\usepackage[explicit]{titlesec}

\titleformat{\chapter}[block]
  {\normalfont\huge\sffamily\textcolor{blueZUT}{#1}}
  {\llap{\color{blueZUT}\huge\thechapter \hspace{1.3ex}{#1}}}
  {0pt}
  {\formatchaptertitle}

\newcommand{\formatchaptertitle}[1]{%
  \parbox[t]{\dimexpr\textwidth-10pt}{\raggedright\LARGE\scshape#1}}
 
%----------------------------------------------------------------------------------------
%	SECTION NUMBERING IN THE MARGIN
%----------------------------------------------------------------------------------------

\makeatletter
\renewcommand{\@seccntformat}[1]{\llap{\textcolor{blueZUT}{\csname the#1\endcsname}\hspace{1em}}}         
%\renewcommand{\chapter}{\@startsection{chapter}{0}{\z@}
%{-4ex \@plus -1ex \@minus -.4ex}
%{10ex \@plus.2ex }
%{\normalfont\huge\sffamily\bfseries\textcolor{blueZUT}}}           
\renewcommand{\section}{\@startsection{section}{1}{\z@}
{-4ex \@plus -1ex \@minus -.4ex}
{1ex \@plus.2ex }
{\normalfont\large\sffamily\bfseries}}
\renewcommand{\subsection}{\@startsection {subsection}{2}{\z@}
{-3ex \@plus -0.1ex \@minus -.4ex}
{0.5ex \@plus.2ex }
{\normalfont\sffamily\bfseries}}
\renewcommand{\subsubsection}{\@startsection {subsubsection}{3}{\z@}
{-2ex \@plus -0.1ex \@minus -.2ex}
{.2ex \@plus.2ex }
{\normalfont\small\sffamily\bfseries}}                        
\renewcommand\paragraph{\@startsection{paragraph}{4}{\z@}
{-2ex \@plus-.2ex \@minus .2ex}
{.1ex}
{\normalfont\small\sffamily\bfseries}}




%----------------------------------------------------------------------------------------
%	MAIN TABLE OF CONTENTS
%----------------------------------------------------------------------------------------

\usepackage{titletoc} % Required for manipulating the table of contents

\contentsmargin{0cm} % Removes the default margin

% Part text styling (this is mostly taken care of in the PART HEADINGS section of this file)
\titlecontents{part}
	[0cm] % Left indentation
	{\addvspace{20pt}\bfseries} % Spacing and font options for parts
	{}
	{}
	{}

% Chapter text styling
\titlecontents{chapter}
	[1.25cm] % Left indentation
	{\addvspace{12pt}\large\sffamily\bfseries\color{blueZUT}} % Spacing and font options for chapters
	{\contentslabel[\Large\thecontentslabel]{1.25cm}} % Formatting of numbered sections of this type
	{} % Formatting of numberless sections of this type
	{\normalsize\;\titlerule*[.5pc]{.}\;\thecontentspage} % Formatting of the filler to the right of the heading and the page number

% Section text styling
\titlecontents{section}
	[1.25cm] % Left indentation
	{\addvspace{3pt}\sffamily\bfseries} % Spacing and font options for sections
	{\contentslabel[\thecontentslabel]{1.25cm}} % Formatting of numbered sections of this type
	{} % Formatting of numberless sections of this type
	{\;\titlerule*[.5pc]{.}\;\thecontentspage} % Formatting of the filler to the right of the heading and the page number

% Subsection text styling
\titlecontents{subsection}
	[1.25cm] % Left indentation
	{\addvspace{1pt}\sffamily\small} % Spacing and font options for subsections
	{\contentslabel[\thecontentslabel]{1.25cm}} % Formatting of numbered sections of this type
	{} % Formatting of numberless sections of this type
	{\ \titlerule*[.5pc]{.}\;\thecontentspage} % Formatting of the filler to the right of the heading and the page number

% Figure text styling
\titlecontents{figure}
	[1.25cm] % Left indentation
	{\addvspace{1pt}\sffamily\small} % Spacing and font options for figures
	{\thecontentslabel\hspace*{1em}} % Formatting of numbered sections of this type
	{} % Formatting of numberless sections of this type
	{\ \titlerule*[.5pc]{.}\;\thecontentspage} % Formatting of the filler to the right of the heading and the page number

% Table text styling
\titlecontents{table}
	[1.25cm] % Left indentation
	{\addvspace{1pt}\sffamily\small} % Spacing and font options for tables
	{\thecontentslabel\hspace*{1em}} % Formatting of numbered sections of this type
	{} % Formatting of numberless sections of this type
	{\ \titlerule*[.5pc]{.}\;\thecontentspage} % Formatting of the filler to the right of the heading and the page number

%----------------------------------------------------------------------------------------
%	HEADERS AND FOOTERS
%----------------------------------------------------------------------------------------
% JK - design and implementation
%----------------------------------------------------------------------------------------
\usepackage{fancyhdr} % Required for header and footer configuration

\pagestyle{fancy} % Enable the custom headers and footers

\renewcommand{\chaptermark}[1]{\markboth{\sffamily\normalsize\bfseries \ #1}{}} % JK - Styling for the current chapter in the header
%\renewcommand{\sectionmark}[1]{\markright{\sffamily\normalsize\thesection\hspace{5pt}#1}{}} % Styling for the current section in the header

\fancyhf{} % Clear default headers and footers

% JK - header with page on the margin and chapter title in the box
\fancyhead[LE,LO]{%
  \textcolor{white}{%
    \llap{%
      \colorbox{blueZUT}{%
        \makebox[6ex][r]{\sffamily\thepage}%
      }%
      \hspace{1.2\marginparsep}%
      \hspace{-\fboxsep}%
    }%
    \textcolor{black}{%
      \colorbox{blueWI!20}{%
        \makebox[\textwidth][l]{\leftmark}}}
  }%
}

\fancypagestyle{plain}{%
   \fancyhead{} % get rid of headers
   \fancyfoot[LE,LO]{
  	\textcolor{white}{%
    	\llap{%
      	\colorbox{blueZUT}{%
        	\makebox[6ex][r]{\sffamily\thepage}%
      }%
      \hspace{1.2\marginparsep}%separation from the margin
      \hspace{-\fboxsep}%
     }%
     }%
    } 
   \renewcommand{\headrulewidth}{0pt} % and the line
}


\renewcommand*{\headrulewidth}{0pt}
\renewcommand*{\footrulewidth}{0pt}


% Removes the header from odd empty pages at the end of chapters
\makeatletter
\renewcommand{\cleardoublepage}{\clearpage\ifodd\c@page\else
\hbox{}
\vspace*{\fill}
\thispagestyle{empty}
\newpage
\fi}

%----------------------------------------------------------------------------------------
%	THEOREM STYLES
%----------------------------------------------------------------------------------------


\newcommand{\intoo}[2]{\mathopen{]}#1\,;#2\mathclose{[}}
\newcommand{\ud}{\mathop{\mathrm{{}d}}\mathopen{}}
\newcommand{\intff}[2]{\mathopen{[}#1\,;#2\mathclose{]}}
\renewcommand{\qedsymbol}{$\blacksquare$}
\newtheorem{notation}{Oznaczenia}[chapter]
\renewcommand{\thmname}{Twierdzenie}

% Boxed/framed environments
\newtheoremstyle{blueZUTnumbox}% Theorem style name
{0pt}% Space above
{0pt}% Space below
{\normalfont}% Body font
{}% Indent amount
{\small\bf\sffamily\color{blueZUT}}% Theorem head font
{\;}% Punctuation after theorem head
{0.25em}% Space after theorem head
{\small\sffamily\color{blueZUT}\thmname{#1}\nobreakspace\thmnumber{\@ifnotempty{#1}{}\@upn{#2}}% Theorem text (e.g. Theorem 2.1)
\thmnote{\nobreakspace\the\thm@notefont\sffamily\bfseries\color{black}---\nobreakspace#3.}} % Optional theorem note

\newtheoremstyle{blacknumex}% Theorem style name
{5pt}% Space above
{5pt}% Space below
{\normalfont}% Body font
{} % Indent amount
{\small\bf\sffamily}% Theorem head font
{\;}% Punctuation after theorem head
{0.25em}% Space after theorem head
{\small\sffamily{\tiny\ensuremath{\blacksquare}}\nobreakspace\thmname{#1}\nobreakspace\thmnumber{\@ifnotempty{#1}{}\@upn{#2}}% Theorem text (e.g. Theorem 2.1)
\thmnote{\nobreakspace\the\thm@notefont\sffamily\bfseries---\nobreakspace#3.}}% Optional theorem note

\newtheoremstyle{blacknumbox} % Theorem style name
{0pt}% Space above
{0pt}% Space below
{\normalfont}% Body font
{}% Indent amount
{\small\bf\sffamily}% Theorem head font
{\;}% Punctuation after theorem head
{0.25em}% Space after theorem head
{\small\sffamily\thmname{#1}\nobreakspace\thmnumber{\@ifnotempty{#1}{}\@upn{#2}}% Theorem text (e.g. Theorem 2.1)
\thmnote{\nobreakspace\the\thm@notefont\sffamily\bfseries---\nobreakspace#3.}}% Optional theorem note

% Non-boxed/non-framed environments
\newtheoremstyle{blueZUTnum}% Theorem style name
{5pt}% Space above
{5pt}% Space below
{\normalfont}% Body font
{}% Indent amount
{\small\bf\sffamily\color{blueZUT}}% Theorem head font
{\;}% Punctuation after theorem head
{0.25em}% Space after theorem head
{\small\sffamily\color{blueZUT}\thmname{#1}\nobreakspace\thmnumber{\@ifnotempty{#1}{}\@upn{#2}}% Theorem text (e.g. Theorem 2.1)
\thmnote{\nobreakspace\the\thm@notefont\sffamily\bfseries\color{black}---\nobreakspace#3.}} % Optional theorem note
\makeatother

% Defines the theorem text style for each type of theorem to one of the three styles above
\newcounter{dummy} 
\numberwithin{dummy}{section}
\theoremstyle{blueZUTnumbox}
\newtheorem{theoremeT}[dummy]{Twierdzenie}
\newtheorem{problem}{Problem}[chapter]
\newtheorem{exerciseT}{\'Cwiczenie}[chapter]
\theoremstyle{blacknumex}
\newtheorem{exampleT}{Przykład}[chapter]
\theoremstyle{blacknumbox}
\newtheorem{definitionT}{Definicja}[section]
\newtheorem{corollaryT}[dummy]{Wniosek}
\theoremstyle{blueZUTnum}
\newtheorem{proposition}[dummy]{Stwierdzenie}

%----------------------------------------------------------------------------------------
%	DEFINITION OF COLORED BOXES
%----------------------------------------------------------------------------------------

\RequirePackage[framemethod=default]{mdframed} % Required for creating the theorem, definition, exercise and corollary boxes

% Theorem box
\newmdenv[skipabove=7pt,
skipbelow=7pt,
backgroundcolor=blueWI!5,
linecolor=blueWI!50,
innerleftmargin=5pt,
innerrightmargin=5pt,
innertopmargin=5pt,
leftmargin=0cm,
rightmargin=0cm,
innerbottommargin=5pt]{tBox}

% Exercise box	  
\newmdenv[skipabove=7pt,
skipbelow=7pt,
rightline=false,
leftline=true,
topline=false,
bottomline=false,
%backgroundcolor=blueWI!10,
linecolor=blueWI,
innerleftmargin=5pt,
innerrightmargin=5pt,
innertopmargin=5pt,
innerbottommargin=5pt,
leftmargin=0cm,
rightmargin=0cm,
linewidth=2pt]{eBox}	

% Definition box
\newmdenv[skipabove=7pt,
skipbelow=7pt,
rightline=false,
leftline=true,
topline=false,
bottomline=false,
linecolor=blueWI,
innerleftmargin=5pt,
innerrightmargin=5pt,
innertopmargin=0pt,
leftmargin=0cm,
rightmargin=0cm,
linewidth=2pt,
innerbottommargin=0pt]{dBox}	

% Corollary box
\newmdenv[skipabove=7pt,
skipbelow=7pt,
rightline=false,
leftline=true,
topline=false,
bottomline=false,
linecolor=gray,
%backgroundcolor=blueWI!5,
linecolor=blueZUT,
innerleftmargin=5pt,
innerrightmargin=5pt,
innertopmargin=5pt,
leftmargin=0cm,
rightmargin=0cm,
linewidth=2pt,
innerbottommargin=5pt]{cBox}

% Creates an environment for each type of theorem and assigns it a theorem text style from the "Theorem Styles" section above and a colored box from above
\newenvironment{theorem}{\begin{tBox}\begin{theoremeT}}{\end{theoremeT}\end{tBox}}
\newenvironment{exercise}{\begin{eBox}\begin{exerciseT}}{\hfill{\color{blueZUT}\tiny\ensuremath{\blacksquare}}\end{exerciseT}\end{eBox}}				  
\newenvironment{definition}{\begin{dBox}\begin{definitionT}}{\end{definitionT}\end{dBox}}	
\newenvironment{example}{\begin{exampleT}}{\hfill{\tiny\ensuremath{\blacksquare}}\end{exampleT}}		
\newenvironment{corollary}{\begin{cBox}\begin{corollaryT}}{\end{corollaryT}\end{cBox}}	

%----------------------------------------------------------------------------------------
%	REMARK ENVIRONMENT
%----------------------------------------------------------------------------------------

\newenvironment{remark}{\par\vspace{10pt}\small % Vertical white space above the remark and smaller font size
\begin{list}{}{
%\leftmargin=35pt % Indentation on the left
\rightmargin=25pt}\item\ignorespaces % Indentation on the right
\makebox[-2.5pt]{\begin{tikzpicture}[overlay]
\node[draw=blueWI!60,line width=1pt,circle,fill=blueWI!25,font=\sffamily\bfseries,inner sep=2pt,outer sep=0pt] at (-15pt,0pt){\textcolor{blueZUT}{U}};\end{tikzpicture}} % Orange R in a circle
\advance\baselineskip -1pt}{\end{list}\vskip5pt} % Tighter line spacing and white space after remarkv

%----------------------------------------------------------------------------------------
%	LISTING ENVIRONMENT
%----------------------------------------------------------------------------------------


\usepackage{listings}

\renewcommand{\lstlistingname}{\small\sffamily\bfseries\color{blueZUT} Algorytm} % Change default listing caption to Algorthm
\renewcommand{\lstlistlistingname}{Lista \lstlistingname ów}

\lstdefinestyle{mystyle}{
    backgroundcolor=\color{grayZUT!10},   
    commentstyle=\color{gray},
    keywordstyle=\color{purple},
    numberstyle=\tiny\color{grayZUT!80},
    stringstyle=\color{red},
    basicstyle=\footnotesize\sffamily,
    breakatwhitespace=true,         
    breaklines=true,                 
    captionpos=b,                    
    keepspaces=false,                 
    numbers=left,                    
    numbersep=5pt,                  
    showspaces=false,                
    showstringspaces=false,
    showtabs=true,                  
    tabsize=2,
    frame=leftline,
    rulecolor = \color{blueWI}, 
    captionpos=t,
}
 
\lstset{style=mystyle}

%----------------------------------------------------------------------------------------
% CAPTIONS
%----------------------------------------------------------------------------------------
% JK - design and implementation
%----------------------------------------------------------------------------------------

\usepackage{caption}
\captionsetup[figure]{name={\small\sffamily\bfseries\color{blueZUT} Rys.}}
\captionsetup[table]{name={\small\sffamily\bfseries\color{blueZUT} Tabela}}
\captionsetup{font={small,sf,singlespacing}}


%----------------------------------------------------------------------------------------
%	HYPERLINKS IN THE DOCUMENTS
%----------------------------------------------------------------------------------------

\usepackage{hyperref}
\hypersetup{hidelinks,backref=true,pagebackref=true,hyperindex=true,colorlinks=false,breaklinks=true,urlcolor=blueZUT,bookmarks=true,bookmarksopen=false}

\usepackage{bookmark}
\bookmarksetup{
open,
numbered,
addtohook={%
\ifnum\bookmarkget{level}=0 % chapter
\bookmarksetup{bold}%
\fi
\ifnum\bookmarkget{level}=-1 % part
\bookmarksetup{color=blueZUT,bold}%
\fi
}
}
 % Insert the commands.tex file which contains the majority of the structure behind the template

%%%%%%%%%%%%%%%%%%%%%%%%%%%%%%%%%%%%%%%%%
% Wydział Informatyki ZUT 
% LaTeX Template
% Praca dyplomowa inżynierska/magisterska
%
% Version 1.0 (10/01/2019)
% Modification assigned by %JK
%
% License:
% CC BY-NC-SA 3.0 (http://creativecommons.org/licenses/by-nc-sa/3.0/)	
%%%%%%%%%%%%%%%%%%%%%%%%%%%%%%%%%%%%%%%%%


\def\HRule{\color{blueWI} \rule{\linewidth}{0.6pt}} % horisontal rule in ZUT color

\def\degreename{praca dyplomowa inżynierska} %alternatywnie {Praca inżynierska}

\def\ttitle{Oprogramowanie do śledzenia ruchu piłkarzy na boisku na podstawie obserwacji przez ruchomą kamerę.} %temat pracy
\def\ttitleEng{	A software for the tracking of football players on the playfield captured by a moving camera} %temat pracy w j. angielskim
\def\datetitle{10.10.2019}
\def\datesubmit{10.10.2050}
\def\placesubmit{Szczecin}

\def\authornames{Karol Działowski} %imię i nazwisko autora
\def\albumno{39259}
\def\speciality{Systemy komputerowe i oprogramowanie}
\def\field{Informatyka}
\def\studyform{studia stacjonarne}

\def\supname{dr inż. Wojciech Sałabun} %imię i nazwisko promotora
%\def\departmentname{Katedra Metod Sztucznej Inteligencji i Matematyki Stosowanej} %nazwa katedry promotora

  % JK - additional definitions used mostly as a content of the title page
% The file MUST be changed by every student/author

%\hypersetup{pdftitle={Title},pdfauthor={Author}} % Uncomment and fill out to include PDF metadata for the author and title of the book


%----------------------------------------------------------------------------------------

\begin{document}

%----------------------------------------------------------------------------------------
%	TITLE PAGE
%----------------------------------------------------------------------------------------
% JK - design and implementation
%----------------------------------------------------------------------------------------

\begingroup
\sffamily  %Set sans serif fonts for title page
\centering %Center all paragraphs
\thispagestyle{empty} % Suppress headers and footers on the title page

%  University  logotype
\includegraphics{ZUT-long.pdf}\\[2cm]

% Thesis title inside two horizontal lines
  \noindent {\HRule }\\[.5cm] % Horizontal line
  {\color{blueZUT} {\Large\ttitle }}\\[.5cm]% Thesis title
  {\Large \ttitleEng }\\[.2cm]% Thesis title in English
  {\HRule} \\[1cm] % Horizontal line

% Thesis type
{\degreename}\\[2cm] 

% Author part
{\color{blueZUT} {\Large\authornames}} \\
{Nr albumu:} {\albumno}\\
{\field}\\
{\speciality}\\
{\studyform}\\[1cm]

% Logo wydziału
\includegraphics[scale=0.5]{WIZUT.pdf}\\[1cm]

% Supervisor part
{napisana pod kierunkiem:} \\
{\textbf\supname}\\
%{\departmentname}


% AUTHOR AND SUPERVISOR parts in  two column designed a s minipages
%
%\begin{minipage}{0.45\textwidth}
%\begin{flushleft}
%\emph{Autor:}\\
%{\textbf\authornames} \\
%{Nr albumu:} {\albumno}\\
%{\field}\\
%{\speciality}\\
%{\studyform}\\
%\end{flushleft}
%\end{minipage}
%\begin{minipage}{0.45\textwidth}
%\begin{flushright} 
%\emph{Promotor:} \\
%{\textbf\supname}\\
%{\departmentname}
%\end{flushright}
%\end{minipage}\\

~\vfill
Data wydania tematu pracy: \datetitle \\
Data złożenia pracy: \datesubmit\\
Miejsce złożenia pracy: \placesubmit\\

\endgroup

%----------------------------------------------------------------------------------------
% EMPTY PAGE AFTER TITLE
%----------------------------------------------------------------------------------------
% JK - design and implementation
%----------------------------------------------------------------------------------------
\newpage
\thispagestyle{empty}
~\vfill

%----------------------------------------------------------------------------------------
% STATEMENT PAGE
%----------------------------------------------------------------------------------------
% JK - design and implementation
%----------------------------------------------------------------------------------------
\newpage
\thispagestyle{empty}


\begin{center}\noindent { \sffamily \large \textbf{OŚWIADCZENIE\\
AUTORA PRACY DYPLOMOWEJ}}\\[1cm] \end{center}

\noindent Oświadczam, że \degreename {\ }pn.{\ }{\emph \ttitle}{\ }
napisana pod kierunkiem \supname  {\ }
jest w całości moim samodzielnym autorskim opracowaniem sporządzonym przy wykorzystaniu wykazanej w pracy literatury przedmiotu i materiałów źródłowych. 
Złożona w dziekanacie Wydziału Informatyki
treść  mojej pracy dyplomowej w formie elektronicznej jest zgodna z treścią w formie pisemnej.\\[.2cm]

\noindent Oświadczam ponadto, że złożona w dziekanacie praca dyplomowa ani jej fragmenty nie były wcześniej przedmiotem procedur procesu dyplomowania związanych z uzyskaniem tytułu zawodowego w uczelniach wyższych.\\[2cm]

\noindent \textit{Podpis autora:}\\[2cm] % Printing/edition date


\noindent \textit{Szczecin, dnia:} % Printing/edition date

%----------------------------------------------------------------------------------------
%	ABSTRACT PAGE
%----------------------------------------------------------------------------------------
%  JK - design and implementation
%----------------------------------------------------------------------------------------
\newpage

\thispagestyle{empty}

\begin{center}\noindent { \sffamily \large \textbf{ STRESZCZENIE}}\\[1cm] \end{center}
\input{streszczenie}%streszczenie.tex file filled with the polish abstract
\vfill
\begin{center}\noindent { \sffamily \large \textbf{ABSTRACT}}\\[1cm] \end{center}
\lipsum[1-2] % Dummy text %abstract.tex file filled with the english abstract
 
%----------------------------------------------------------------------------------------
%	TABLE OF CONTENTS
%----------------------------------------------------------------------------------------



\pagestyle{empty}  % JK - Disable footer and header in TOC
\tableofcontents % Print the table of contents itself
\addtocontents{toc}{\protect\thispagestyle{empty}} % JK - to keep the empty header and footer in TOC

\cleardoublepage % Forces the first chapter to start on an odd page so it's on the right side of the book

\pagestyle{fancy} % Enable headers and footers again


%----------------------------------------------------------------------------------------
%	CHAPTER 1
%----------------------------------------------------------------------------------------

%----------------------------------------------------------------------------------------
%	CHAPTER 1
%----------------------------------------------------------------------------------------

\chapter{Rozdział zawierający tekst}

\section{Podrozdział zawierający tylko tekst}

\lipsum[1-7] % Dummy text

%------------------------------------------------

\section{Cytowania}


To stwierdzenie wymaga cytowania \cite{article_key}; ten jest bardziej szczegółowy \cite[162]{book_key}.

Przykład cytowania źródła online: 
\cite{knuthwebsite}.

{\bf Ważne:} Kompilację źródeł należy wykonać z wykorzystaniem Biber not Bibtex. Pakiet do zarządzania bibliografią to biblatex. Jego konfiguracja jest dostępna w pliku structure.tex

%------------------------------------------------

\section{Lists}

Listy są przydatne do przedstawiania informacji w sposób zwięzły i / lub uporządkowany \footnote{Przykład przypisu dolnego.}.

\subsection{Lista porządkowana}

\begin{enumerate}
\item Element 1
\item Element 2
\item Element 34234
\end{enumerate}

\subsection{Lista wypunktowana}

\begin{itemize}
\item Element 1
\item Element 2
\item Element 34234
\end{itemize}

\subsection{Lista z nagłówkami (descriptions-definitions)}

\begin{description}
\item[Nazwa] Opis
\item[Słowo] Definicja
\item[Komentarz] Wywód
\end{description}



%----------------------------------------------------------------------------------------
%	CHAPTER 2
%----------------------------------------------------------------------------------------

%----------------------------------------------------------------------------------------
%	CHAPTER 2
%----------------------------------------------------------------------------------------

\chapter{Elementy zawierane w tekście}

\section{Twierdzenia}
Poniżej znajduje się przykład twierdzenia.

\subsection{Kilka wzorów}
A oto twierdzenie składające się z kilku równań.

\begin{theorem}[Tytuł twierdzenia]
In $E=\mathbb{R}^n$ all norms are equivalent. It has the properties:
\begin{align}
& \big| ||\mathbf{x}|| - ||\mathbf{y}|| \big|\leq || \mathbf{x}- \mathbf{y}||\\
&  ||\sum_{i=1}^n\mathbf{x}_i||\leq \sum_{i=1}^n||\mathbf{x}_i||\quad\text{where $n$ is a finite integer}
\end{align}
\end{theorem}

\subsection{Twierdzenie jednolinijkowe}
Jest to twierdzenie składające się z tylko jednej linii.

\begin{theorem}
Zbiór $\mathcal{D}(G)$ ma gęstość $L^2(G)$, $|\cdot|_0$. 
\end{theorem}

%------------------------------------------------

\section{Definicje}

To jest przykład definicji. Definicja może być matematyczna lub może definiować koncepcję.

\begin{definition}[Tytuł definicji]
Given a vector space $E$, a norm on $E$ is an application, denoted $||\cdot||$, $E$ in $\mathbb{R}^+=[0,+\infty[$ such that:
\begin{align}
& ||\mathbf{x}||=0\ \Rightarrow\ \mathbf{x}=\mathbf{0}\\
& ||\lambda \mathbf{x}||=|\lambda|\cdot ||\mathbf{x}||\\
& ||\mathbf{x}+\mathbf{y}||\leq ||\mathbf{x}||+||\mathbf{y}||
\end{align}
\end{definition}

%------------------------------------------------

\section{Oznaczenia}

\begin{notation}
Given an open subset $G$ of $\mathbb{R}^n$, the set of functions $\varphi$ are:
\begin{enumerate}
\item Bounded support $G$;
\item Infinitely differentiable;
\end{enumerate}
a vector space is denoted by $\mathcal{D}(G)$. 
\end{notation}

%------------------------------------------------

\section{Uwagi}

To jest przykład uwagi.

\begin{remark}
Przedstawione tutaj koncepcje znajdują się obecnie w powszechnym zastosowaniu w matematyce. Vector spaces are taken over the field $\mathbb{K}=\mathbb{R}$, however, established properties are easily extended to $\mathbb{K}=\mathbb{C}$.
\end{remark}

%------------------------------------------------

\section{Corollaries}

To jest przykład następstwa.

\begin{corollary}[Corollary name]
The concepts presented here are now in conventional employment in mathematics. Vector spaces are taken over the field $\mathbb{K}=\mathbb{R}$, however, established properties are easily extended to $\mathbb{K}=\mathbb{C}$.
\end{corollary}

%------------------------------------------------

\section{Stwierdzenia}

To jest przykład stwierdzenia

\subsection{Kilka równań}

\begin{proposition}[Proposition name]
It has the properties:
\begin{align}
& \big| ||\mathbf{x}|| - ||\mathbf{y}|| \big|\leq || \mathbf{x}- \mathbf{y}||\\
&  ||\sum_{i=1}^n\mathbf{x}_i||\leq \sum_{i=1}^n||\mathbf{x}_i||\quad\text{where $n$ is a finite integer}
\end{align}
\end{proposition}

\subsection{Jednolinijkowe}

\begin{proposition} 
Let $f,g\in L^2(G)$; if $\forall \varphi\in\mathcal{D}(G)$, $(f,\varphi)_0=(g,\varphi)_0$ then $f = g$. 
\end{proposition}

%------------------------------------------------

\section{Przykłady}

To jest przykład przykładów.

\subsection{Równania i tekst}

\begin{example}
Let $G=\{x\in\mathbb{R}^2:|x|<3\}$ and denoted by: $x^0=(1,1)$; consider the function:
\begin{equation}
f(x)=\left\{\begin{aligned} & \mathrm{e}^{|x|} & & \text{si $|x-x^0|\leq 1/2$}\\
& 0 & & \text{si $|x-x^0|> 1/2$}\end{aligned}\right.
\end{equation}
The function $f$ has bounded support, we can take $A=\{x\in\mathbb{R}^2:|x-x^0|\leq 1/2+\epsilon\}$ for all $\epsilon\in\intoo{0}{5/2-\sqrt{2}}$.
\end{example}

\subsection{Akapit tekstu}

\begin{example}[Pewien przykład]
\lipsum[2]
\end{example}

%------------------------------------------------

\section{Ćwiczenia}

To jest przykład ćwiczenia.

\begin{exercise}
This is a good place to ask a question to test learning progress or further cement ideas into students' minds.
\end{exercise}

\begin{exercise}
This is a good place to ask a question to test learning progress or further cement ideas into students' minds.
\end{exercise}

%------------------------------------------------

\section{Problems}

\begin{problem}
Jaka jest średnia prędkość lotu pustej jaskółki?
\end{problem}

%------------------------------------------------

\section{Kod źródłowy}

\lipsum[1]

\begin{lstlisting}[language=Python, caption=Fragment algorytmu xxx]

import numpy as np
 
def incmatrix(genl1,genl2):
    m = len(genl1)
    n = len(genl2)
    M = None #to become the incidence matrix
    VT = np.zeros((n*m,1), int)  #dummy variable
 
    #compute the bitwise xor matrix
    M1 = bitxormatrix(genl1)
    M2 = np.triu(bitxormatrix(genl2),1) 
 
    for i in range(m-1):
        for j in range(i+1, m):
            [r,c] = np.where(M2 == M1[i,j])
            for k in range(len(r)):
                VT[(i)*n + r[k]] = 1;
                VT[(i)*n + c[k]] = 1;
                VT[(j)*n + r[k]] = 1;
                VT[(j)*n + c[k]] = 1;
 
                if M is None:
                    M = np.copy(VT)
                else:
                    M = np.concatenate((M, VT), 1)
 
                VT = np.zeros((n*m,1), int)
                String = "Alice in the Wondeland"
 
    return M
\end{lstlisting}

Algortym \ref{codeyyy} został.....

\begin{lstlisting}[language=Python, caption=Fragment algorytmu yyy, label=codeyyy]

import numpy as np
 
def incmatrix(genl1,genl2):
    m = len(genl1)
    n = len(genl2)
    M = None #to become the incidence matrix
    VT = np.zeros((n*m,1), int)  #dummy variable
 
    #compute the bitwise xor matrix
    M1 = bitxormatrix(genl1)
    M2 = np.triu(bitxormatrix(genl2),1) 
 
\end{lstlisting}


%----------------------------------------------------------------------------------------
%	CHAPTER 3
%----------------------------------------------------------------------------------------
\chapter{Wizualizacje}

\section{Tabele w pracy}


%\color{blueWI}\rule
Paragraph
\begin{table}[h]
\centering
\caption{Tabelka z wynikami} % JK- Tabela podpisana z góry
\begin{tabular}{l l l}
\toprule
\textbf{Treatments} & \textbf{Response 1} & \textbf{Response 2}\\
\midrule
Treatment 1 & 0.0003262 & 0.562 \\
Treatment 2 & 0.0015681 & 0.910 \\
Treatment 3 & 0.0009271 & 0.296 \\
\bottomrule
\end{tabular}
\label{tab:example} % Unique label used for referencing the table in-text
%\addcontentsline{toc}{table}{Table \ref{tab:example}} % Uncomment to add the table to the table of contents
\end{table}

Odwołanie do Tabeli z automatycznym numerowaniem: \ref{tab:example}.

%------------------------------------------------

\section{Ilustarcje w pracy}

\begin{figure}[h]
\centering\includegraphics[scale=0.5]{placeholder.jpg}
\caption{Pole z miejscem na rysunek}
\label{fig:placeholder} % Unique label used for referencing the figure in-text
%\addcontentsline{toc}{figure}{Figure \ref{fig:placeholder}} % Uncomment to add the figure to the table of contents
\end{figure}

Odwołanie do Rysunku z automatycznym numerowaniem \ref{fig:placeholder}.

\lipsum[1-7] % Dummy text

%----------------------------------------------------------------------------------------
%	BIBLIOGRAPHY
%----------------------------------------------------------------------------------------
%  JK - design and implementation
%----------------------------------------------------------------------------------------

\chapter*{Spis literatury}
\addcontentsline{toc}{chapter}{\textcolor{blueZUT}{Spis literatury}} % Add a Bibliography heading to the table of contents


% Filters for diferent types of bib entries

\defbibfilter{literatura}{
  type=article or 
  type=book or
  type=inproceedings
}

\defbibfilter{linki}{
  type=url or
  type=online 
}

%------------------------------------------------
% Two parts of literature - printed and online

\section*{Artykuły i podręczniki}
\addcontentsline{toc}{section}{Artykuły i podręczniki}
\printbibliography[heading=bibempty,filter=literatura]

%------------------------------------------------

\section*{Źródła internetowe}
\addcontentsline{toc}{section}{Źródła internetowe}
\printbibliography[heading=bibempty,filter=linki]

%----------------------------------------------------------------------------------------

\end{document}
