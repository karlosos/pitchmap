%----------------------------------------------------------------------------------------
%	CHAPTER 1
%----------------------------------------------------------------------------------------

\chapter{Rozdział zawierający tekst}

\section{Podrozdział zawierający tylko tekst}

\lipsum[1-7] % Dummy text

%------------------------------------------------

\section{Cytowania}


To stwierdzenie wymaga cytowania \cite{article_key}; ten jest bardziej szczegółowy \cite[162]{book_key}.

Przykład cytowania źródła online: 
\cite{knuthwebsite}.

{\bf Ważne:} Kompilację źródeł należy wykonać z wykorzystaniem Biber not Bibtex. Pakiet do zarządzania bibliografią to biblatex. Jego konfiguracja jest dostępna w pliku structure.tex

%------------------------------------------------

\section{Lists}

Listy są przydatne do przedstawiania informacji w sposób zwięzły i / lub uporządkowany \footnote{Przykład przypisu dolnego.}.

\subsection{Lista porządkowana}

\begin{enumerate}
\item Element 1
\item Element 2
\item Element 34234
\end{enumerate}

\subsection{Lista wypunktowana}

\begin{itemize}
\item Element 1
\item Element 2
\item Element 34234
\end{itemize}

\subsection{Lista z nagłówkami (descriptions-definitions)}

\begin{description}
\item[Nazwa] Opis
\item[Słowo] Definicja
\item[Komentarz] Wywód
\end{description}

